\documentclass[11pt]{article}
\usepackage[utf8]{inputenc} % standard input encoding
\usepackage[T1]{fontenc} % standard modern font encoding
\usepackage{lmodern} % use Latin Modern font (it has vectorized special characters = better copying od searching in pdfs)
\usepackage[czech]{babel}
\usepackage[a4paper]{geometry} % margin adjustment
\usepackage[pdfusetitle]{hyperref} % metadata, content in pdf, hyperlinks

% bibliography
\usepackage[backend=bibtex]{biblatex}
\addbibresource{bibliography.bib}

\def\MainTitle{Automatická detekce ischemické léze z MR obrazových dat u cévní mozkové příhody}
\def\Subtitle{Diplomový projekt}
\date{\large \hfill \today}

\hypersetup{pdftitle={\MainTitle}} % add main title to the pdf metadata
\title{
	\ifdefined\Subtitle \large \Subtitle \\[1em] \fi
	\LARGE \textbf{\MainTitle} \\[2em]
	\begin{large}
	\begin{minipage}{3cm}
		\textbf{Autor:}\\
		Jakub Šmíd
	\end{minipage}
	\hfill
	\begin{minipage}{6cm}
		\textbf{Vedoucí:}\\
		doc. MUDr. Jakub Otáhal, Ph.D. \\
		Ing. David Kala
	\end{minipage}
	\end{large}
}

\begin{document}
\maketitle

\section{Popis projektu}
Cévní mozková příhoda (mrtvice) je jedním z nejčastějších onemocnění a příčin úmrtí celosvětově. Zásadní krok při jejím vyšetření je segmentace poškozené tkáně na obrazech z magnetické resonance. V současné klinické praxi se segmentace provádí manuálním obkreslováním a kvůli tomu je velmi časově náročná a výsledky podléhají značné subjektivitě. Cílem této práce je celý proces zrychlit přidáním prvků automatické segmentace obrazu.
Práce bude probíhat v úzké spolupráci s Fakultní nemocnicí v Motole a neurovědci z EpiReC zabývající se dlouhodobě problematikou cévní mozkové příhody a souvisejících onemocnění.

Téma propojuje technické znalosti (zpracování obrazu, návrh algoritmů) s lékařským prostředím neurověd.

\section{Dataset}
\begin{itemize}
	\item FLAIR vs DWI
	\item Nifty - popis, registrace
	\item Možnosti rozšíření datasetu (stažení dalších dat, augmentace)
\end{itemize}

\section{Přehledové studie}

\section{Ztrátové funkce a hodnocení úspěšnosti segmentace}

\section{Architektury neuronových sítí}
\begin{itemize}
	\item ISLES
	\item 2D vs 3D segmentace
\end{itemize}

\printbibliography

\end{document}
